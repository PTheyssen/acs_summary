\section{Communication}

\paragraph{Learning goals}
\begin{itemize}
\item approaches to design \textbf{communication abstractions}
  \begin{itemize}
  \item transient vs. persistent
  \item synchronous vs. asynchronous
  \end{itemize}

\item Design and implementation of
  \textbf{message-oriented middleware} (MOM)

\item organizing systems using \textbf{BASE} methodology

\item relationship BASE to eventual consistency and
  \textbf{CAP theorem} (it is impossible to provide
  \textbf{C}onsistency, \textbf{A}vailability and
  \textbf{P}artition Tolerance at the same time in distributed systems)

\item alternative communication abstractions
  \textbf{data streams} and \textbf{multicast / gossip}
\end{itemize}


\paragraph{Partitioning}
\begin{itemize}
\item use independent services to store different data
\item Scalability and Availability improves but now coordination
  necessary
\item employ a communication abstraction
\end{itemize}

\paragraph{Recall: Protocols and Layering in the Internet}
\begin{itemize}
\item Layering
  \begin{itemize}
  \item System broken into vertical hierarchy of protocols
  \item Service provided by one layer based solely
    on service provided by layer below
  \end{itemize}
\item Internet model (here four layers)
  \begin{itemize}
  \item \textbf{Application} (e.g. HTTP, DNS, Email..)
  \item \textbf{Transport} (TCP, UDP)
  \item \textbf{Network} (IP)
  \item \textbf{Link} (Ethernet)
  \end{itemize}
\end{itemize}

\paragraph{Recall: Layers in Hosts and Routers}
\begin{itemize}
\item Link and network layers implemented everywhere
\item End-to-end layer (i.e., transport and application)
  implemented only at hosts
\end{itemize}

\paragraph{Different Type of Communication}
\begin{itemize}
\item Asynchronous: sender can immediately return after sending message
  without waiting for any acknowledgments from other components

\item Synchronous
  \begin{itemize}
  \item Synchronize at request submission (1)
  \item Synchronize at request delivery (2)
  \item Synchronize after being fully processed by recipient (3)
  \end{itemize}

\item \textbf{Transient} vs. \textbf{persistent} (temporal decoupling):
  in persistent communication the sender can send a message
  and go offline and the receiver can go online at any time and
  gets the message that was send to him. Sender and receiver
  can be independent in terms of operation time.
\end{itemize}

Examples:
\begin{itemize}
\item \textbf{Persistent} and \textbf{Asynchronous}: Email

\item \textbf{Persistent} and \textbf{Synchronous}:
  Message-queuing systems (1), \\
  Online Chat (1) + (2)

\item \textbf{Transient} and \textbf{Asynchronous}: UDP, Erlang

\item \textbf{Transient} and \textbf{Synchronous}:
  Asynchronous RPC (2), RPC (3)
\end{itemize}

\paragraph{RPC}
\begin{itemize}
\item Transparent and hidden communication
\item Synchronous
\item Transient
\end{itemize}

\paragraph{Message-Oriented}
\begin{itemize}
\item Explicit communication SEND/RECEIVE of point-to-point
  messages
\item Synchronous vs. Asynchronous
\item Transient vs. Persistent
\end{itemize}

\paragraph{Message-Oriented Persistent Communication}
\begin{itemize}
\item Queues make sender and receiver loosely-coupled
\item Modes of execution of sender/receiver
  \begin{itemize}
  \item both running
  \item Sender running, Receiver passive
  \item Sender passive, Receiver running
  \item both passive
  \end{itemize}
\end{itemize}

\paragraph{Queue Interface}
\begin{itemize}
\item Put: Put message in queue
\item Get: Remove first message from queue (blocking)
\item Poll: Check for message and remove first (non-blocking)
\item Notify: Handler that is called when message is added
\end{itemize}

Source / destination are decoupled by \textbf{queue names}

Multiple nodes (distributed system) inside message queuing
system for high throughput communication
\begin{itemize}
\item \textbf{Relays:} store and forward messages
\item \textbf{Brokers:} gateway to transform message formats
\end{itemize}

\paragraph{How to Employ Queues to Decouple System}
Example Bank Transfer
\begin{itemize}
\item for scalability partition accounts onto different computers
\item use message queue between two accounts
\item for ensuring \textbf{atomicity} we need to use
  commit protocol (two-phase commit)
\end{itemize}


\paragraph{The CAP theorem}
\textbf{CAP Theorem:} A scalable service cannot achieve all three
properties simultaneously
\begin{itemize}
\item Consistency: (i.e. atomicity) client perceives set of
  operations occurred all at once
\item Availability: every request must result in an intended
  response
\item Partition tolerance: operations terminate,
  even if the network is partitioned
\end{itemize}

\paragraph{ACID}
\begin{itemize}
\item Using 2PC (two phase commit) guarantees atomicity (C in CAP)
\item If 2PC is used, a transaction is not guaranteed to complete
  in the case of network partition (either Abort or Blocked)
  $\rightarrow$ we choose Consistency over Availability
\item If an application can benefit from choosing A over C
  we need different method (BASE). For example in a social network
  its more important to have availability.
\end{itemize}

\paragraph{BASE}
\begin{itemize}
\item \textbf{Basically-Available:} only components affected by failure
  become unavailable, not whole system
\item \textbf{Soft-State:} a component's state may be out-of-date,
  and events may be lost without affecting availability
\item \textbf{Eventually Consistent:} under no further updates and no
  failures, partitions converge to consistent state
\end{itemize}

\paragraph{A BASE Scenario}
\begin{itemize}
\item Users buy and sell items
\item simple transaction for item exchange
\end{itemize}

Now we can \textbf{Decouple Item Exchange with Queues},
by having one component taking care of transactions and
the other one of users and updates to them. The following
issued arise
\begin{itemize}
\item Tolerance to loss
\item Idempotence
\item Order: order can be implemented by a last transaction
  pointer in the receiver, would be to restrictive on queue implementation
  to ensure a order
\end{itemize}

\paragraph{Other Types of Communication Abstractions}
\begin{itemize}
\item \textbf{Stream-Oriented}
  \begin{itemize}
  \item Continuous vs. discrete
  \item Asynchronous vs. Synchronous vs. Isochronous
  \item Simple vs. complex
  \end{itemize}
\item \textbf{Multicast}
  \begin{itemize}
  \item SEND/RECEIVE over groups
  \item Application-level multicast vs. gossip
  \end{itemize}
\end{itemize}

\paragraph{Gossip}
\begin{itemize}
\item Epidemic protocols
  \begin{itemize}
  \item No central coordinator
  \item nodes with new information are infected, and try to spread
    information
  \end{itemize}

\item Anti-entropy approach
  \begin{itemize}
  \item Each node communicates with random node
  \item Round: every node does the above
  \item Pull vs. push vs. both
  \item Spreading update form single to all nodes takes
    O(log(N))
  \end{itemize}

\item \textbf{Gossiping}
  \begin{itemize}
  \item more similar to real world analogy
  \item If P is updated it tells random Q
  \item if Q already knows, P can lose interest
    with some percentage
  \item at some threshold P stops telling random nodes
  \item Not guaranteed that all nodes get infected by update
  \end{itemize}
\end{itemize}




% LocalWords:  middleware multicast onsistency vailability artition
% LocalWords:  Scalability DNS TCP UDP RPC scalability atomicity
% LocalWords:  scalable Idempotence Isochronous
